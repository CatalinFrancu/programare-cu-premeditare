\chapter{Arbori de intervale}

Arborii de intervale\footnote{Există o inversiune între nomenclatura internațională și cea românească. Internațional, structura pe care o învățăm astăzi se numește \href{https://cp-algorithms.com/data_structures/segment_tree.html}{segment tree}, iar \href{https://en.wikipedia.org/wiki/Interval_tree}{interval tree} este o structură diferită, care stochează colecții de intervale. Cîțiva ani am înotat împotriva curentului și am fost (posibil) singurul român care se referea la această structură ca „arbori de segmente”. În acest curs am adoptat și eu denumirea încetățenită.
} (AINT) sînt o structură foarte puternică și flexibilă. Ușurința implementării depinde de natura operațiilor pe care dorim să le admitem.

\section{Reprezentare}

Ca multe alte structuri (heap-uri, AIB, păduri disjuncte), arborii de intervale se reprezintă pe un simplu vector. Ei sînt arbori doar la nivel logic, în sensul că fiecare poziție din vector are o altă poziție drept părinte.

Pentru început, să presupunem că vectorul dat are $n = 2^k$ elemente. Atunci vectorul necesar $S$ are $2n$ elemente, în care cele $n$ elemente date sînt stocate începînd cu poziția $n$. Apoi,

\begin{itemize}
  \item Cele $n/2$ elemente anterioare stochează valori agregate (sume, minime, xor etc.) pentru perechi de valori din vectorul dat.
  \item Cele $n/4$ elemente anterioare stochează valori agregate pentru grupe de 4 valori din vectorul dat.
  \item $\dots$
  \item Elementul $S[1]$ stochează valoarea agregată a întregului vector.
  \item Valoarea $S[0]$ rămîne nefolosită.
\end{itemize}

Iată un exemplu pentru $n = 16$. Datele de la intrare se regăsesc pe pozițiile 16-31.

\import{./figures}{segment-tree-representation.tex}

Facem cîteva observații preliminare:

\begin{itemize}
  \item Fiii unui nod $i$ sînt $2i$ și $2i + 1$.
  \item Părintele lui $i$ este $\lfloor i / 2 \rfloor$.
  \item Toți fiii stîngi au numere pare și toți fiii drepți au numere impare.
\end{itemize}

De exemplu, fiii lui 6 sînt 12 și 13, iar fiii acestora sînt respectiv 24-25 și 26-27. Aceasta corespunde cu intenția noastră ca 6 stocheze informații agregate (suma) despre nodurile 24-27.

După cum vom vedea în secțiunea următoare, arborii de intervale obțin timpi logaritmici pentru operații, deoarece numărul de niveluri este $\log n$.

\subsection{Memoria necesară}

În această formă, structura necesită $2n$ memorie pentru $n$ elemente dacă $n$ este putere a lui 2 sau foarte aproape. De exemplu, pentru $n = 1024$, sînt necesare $2048$ de celule. Dar, dacă $n$ depășește cu puțin o putere a lui $2$, atunci el trebuie rotunjit în sus. Pentru $n = 1025$, baza arborelui necesită $2048$ de celule, iar arborele în întregime necesită $4096$ de celule. De aceea spunem că, în cel mai rău caz, arborele poate ajunge la $4n$ celule ocupate în cel mai rău caz.

În realitate, necesarul este doar de $3n$ cu puțină atenție la alocare. Pentru $n = 1025$, alocăm $2048$ de celule pentru nivelurile superioare ale arborelui, dar putem aloca fix $1025$ pentru bază (nu $2048$). Totalul este circa $3n$.

Pentru a calcula următoare putere a lui 2, putem folosi bucla naivă:

\begin{minted}{c}
int p = 1;
while (p < n) {
  p *= 2;
}
n = p;
\end{minted}

Sau o buclă care folosește \textit{bit hacks}:

\begin{minted}{c}
while (n & (n - 1)) {
  n += n & -n;
}
\end{minted}

Mai concis, putem folosi funcția \mintinline{c}{__builtin_clz(x)}, care ne spune cu cîte zerouri începe numărul \mintinline{c}{x}:

\begin{minted}{c}
n = 1 << (32 - __builtin_clz(n - 1));
\end{minted}

\subsection{Reprezentări alternative}

Există și reprezentări mai compacte, care ocupă exact $2n-1$ noduri, adică strictul necesar teoretic. Iată un exemplu pentru un vector cu 6 noduri.

\import{./figures}{segment-tree-2n-nodes.tex}

Vedem că frunzele (adică vectorul dat, $v[0] \dots v[5]$) se află pe pozițiile consecutive 6-11. În schimb, această reprezentare pare mai greu de vizualizat și încalcă o abstracție importantă: frunzele nu mai sînt la același nivel. Structura se pretează la operațiile de modificare și interogare, dar nu sînt sigur că se pretează și la restul operațiilor pe care le discutăm în secțiunile următoare. De aceea prefer să folosesc și să predau structura rotunjită la $2^k$ noduri.

\section{Operații elementare}

\subsection{Actualizarea punctuală}

Nu uitați că poziția $i$ din datele de intrare este stocată efectiv în $s[n + i]$. Apoi, cînd elementul aflat pe poziția $i$ primește valoarea $val$, toate nodurile care acoperă poziția $i$ trebuie recalculate:

\begin{minted}{c}
void set(int pos, int val) {
  pos += n;
  s[pos] = val;
  for (pos /= 2; pos; pos /= 2) {
    s[pos] = s[2 * pos] + s[2 * pos + 1];
  }
}
\end{minted}

Dacă nu ni se dă noua valoare absolută, ci variația $\delta$ față de valoarea anterioară, atunci codul este chiar mai simplu, căci toți strămoșii poziției se modifică tot cu $\delta$:

\begin{minted}{c}
void add(int pos, int delta) {
  for (pos += n; pos; pos /= 2) {
    s[pos] += delta;
  }
}
\end{minted}

Apropo de \textit{clean code}: Remarcați că am denumit funcțiile \mintinline{c}{set} și \mintinline{c}{add}, nu le-am denumit pe ambele \mintinline{c}{update}. Astfel am evidențiat diferența dintre ele.

\subsection{Construcția în \texorpdfstring{$\bigoh(n \log n)$}{O(n log n)}}

O variantă de construcție este să invocăm funcția set de mai sus pentru fiecare valoare de la intrare. Complexitatea va fi $\bigoh(n \log n)$.

\subsection{Construcția în \texorpdfstring{$\bigoh(n)$}{O(n)}}

Putem reduce timpul de construcție dacă doar inserăm valorile frunzelor, fără a le propaga la strămoși. La final calculăm foarte simplu nodurile interne, în ordine descrescătoare.

\begin{minted}{c}
void build() {
  for (int i = n - 1; i >= 1; i--) {
    s[i] = s[2 * i] + s[2 * i + 1];
  }
}
\end{minted}

\subsection{Calculul sumei pe interval}

Să calculăm suma pe intervalul original [2, 12], care corespunde intervalului [18, 28] din reprezentarea internă. Ideea este să descompunem acest interval într-un număr logaritmic de segmente, mai exact [18,19], [20,23], [24,27] și [28,28]. Avantajul descompunerii este că avem deja calculate sumele acestor intervale, respectiv în nodurile 9, 5, 6 și 28.

\import{./figures}{segment-tree-range-sum.tex}

Pornim cu doi pointeri $l$ și $r$ din capetele interogării date. Apoi procedăm astfel:

\begin{itemize}
  \item Dacă $l$ este fiu stîng, putem aștepta ca să includem un strămoș al său, care va include și alte poziții utile. În schimb, dacă $l$ este fiu drept, trebuie să îl includem în sumă, căci orice strămoș al său va include și elemente inutile din stînga lui $l$. Apoi avansăm $l$ spre dreapta.

  \item Printr-un raționament similar, dacă $r$ este fiu stîng, includem valoarea sa în sumă și avansăm $r$ spre stînga.

  \item Urcăm pe nivelul următor prin înjumătățirea lui $l$ și $r$.

  \item Continuăm cît timp $l \leq r$.
\end{itemize}

Astfel, vom selecta cel mult două intervale de pe fiecare nivel al arborelui și vom restrînge corespunzător intervalul dat, pînă cînd îl reducem la zero. De aici rezultă complexitatea logaritmică.

\begin{minted}{c}
long long query(int l, int r) { // [l, r] închis
  long long sum = 0;

  l += n;
  r += n;

  while (l <= r)  {
    if (l & 1) {
      sum += s[l++];
    }
    l >>= 1;

    if (!(r & 1)) {
      sum += s[r--];
    }
    r >>= 1;
  }

  return sum;
}
\end{minted}

Clarificare: la ultimul nivel, dacă $l = r$, atunci $s[l]$ va fi selectat exact o dată, fie datorită lui $l$, fie datorită lui $r$, după cum poziția este impară sau pară.

\subsection{Căutarea unei sume parțiale}

Ca și la AIB-uri, dacă toate valorile sînt pozitive are sens întrebarea: pe ce poziție suma parțială atinge valoarea $P$? Pentru simplitate, recomand să adăugați o santinelă de valoarea infinită pe poziția $n$. Aceasta garantează că suma parțială se atinge întotdeauna, iar dacă răspunsul este $n$, atunci de fapt suma parțială nu există în vectorul fără santinelă.

\begin{minted}{c}
int search(int sum) {
  int pos = 1;

  while (pos < n) {
    pos *= 2;
    if (sum > s[pos]) {
      sum -= s[pos++];
    }
  }

  return pos - n;
}
\end{minted}

\subsection{Căutarea într-un arbore de maxime}

Dat fiind un vector $v$ cu $n$ elemente, ni se cere să răspundem la interogări de tipul $\langle pos, val \rangle$ cu semnificația: găsiți cea mai mică poziție $i > pos$ pe care se află o valoare $v[i] > val$. În secțiunea următoare vom vedea problemele Points și Împiedicat care au această nevoie.

Pentru rezolvare, să construim peste acest vector un arbore de intervale de maxime. Fiecare nod stochează maximul dintre cei doi fii ai săi. Ca urmare, fiecare nod stochează maximul dintre frunzele pe care le subîntinde. Atunci soluția constă din doi pași:

\begin{itemize}
  \item Mergi la dreapta și în sus, similar pointerului $l$ din operația de sumă pe interval prezentată anterior. Oprește-te cînd ajungi la un nod cu o valoare $> val$. Știm că acest nod subîntinde cel puțin o frunză de valoare $> val$.
  \item Din acest nod, coboară în fiul care are la rîndul său o valoare $> val$. Dacă ambii fii au această proprietate, coboară în fiul stîng. Oprește-te cînd ajungi la o frunză.
\end{itemize}

Pentru a simplifica codul, putem adăuga o santinelă infinită la finalul vectorului, ca să ne asigurăm că problema are soluție.

\begin{minted}{c}
int find_first_after(int pos, int val) {
  pos += n + 1;

  while (v[pos] <= val) {
    if (pos & 1) {
      pos++;
    } else {
      pos >>= 1;
    }
  }

  while (pos < n) {
    pos = (v[2 * pos] > val) ? (2 * pos) : (2 * pos + 1);
  }

  return pos - n;
}
\end{minted}

Am inclus acest algoritm, deși este rar întîlnit în pratică, pentru a ilustra flexibilitatea uriașă a arborilor de intervale.

\subsection{Adaptarea la alte tipuri de operații}

Aceeași structură de date poate răspunde la multe alte feluri de actualizări și interogări. Nu detaliem aici, vom studia probleme. Ce este important este să ne dăm seama ce stocăm în fiecare nod și cum combină părintele informațiile din cei doi fii.

\subsection{Implementarea recursivă}

Există și o implementare recursivă, pe care nu o vom discuta acum (o menționez doar ca să o fac de rîs). O vom discuta mai tîrziu în acest capitol. Este păcat că mulți elevi învață și stăpînesc doar acea implementare, pe care o aplică și cînd nu este nevoie de ea, deși implementarea iterativă de mai sus este de 2-3 ori mai rapidă. Implementarea iterativă ar trebui să fie implementarea voastră de referință oricînd este suficientă.

Exemplu: din implementarea iterativă rezultă imediat că:

\begin{enumerate}
  \item  Complexitatea este $\bigoh(\log n)$, întrucît $l$ și $r$ urcă exact un nivel la fiecare iterație.

  \item De pe fiecare nivel selectăm cel mult două intervale.
\end{enumerate}

Vă urez succes să demonstrați aceste lucruri în implementarea recursivă. \emoji{smiling-face-with-horns}
