\chapter{Arbori de intervale cu propagare \textit{lazy}}

\section{Operații pe interval}

Să reluăm exemplul din capitolul trecut și să spunem acum că dorim să adăugăm 100 pe intervalul [2, 12], corespunzător nodurilor [12, 28] din arbore.

Ca să nu facem efort $\bigoh(n)$, vom descompune intervalul ca mai înainte și vom nota informația „+100” în nodurile 9, 5, 6 și 28, cu semnificația că valoarea reală a tuturor frunzelor de sub aceste noduri a crescut cu 100.

\import{./figures}{segment-tree-range-add.tex}

Aceasta este o \textbf{informație \textit{lazy}}: o informație care stă într-un nod intern și care trebuie propagată tuturor frunzelor subîntinse de acel nod. Totuși, amînăm efortul acestei propagări pînă cînd el devine strict necesar; tocmai de aceea se numește \textbf{propagare \textit{lazy}}. (Mulți elevi denumesc întreaga structură „AINT cu \textit{lazy}”, dar asta este... lene.)

Evaluarea \textit{lazy} este un concept des întîlnit:

\begin{itemize}
  \item Memoizarea unor valori într-un vector / matrice, cu speranța că nu va fi nevoie să calculăm tabelul complet.

  \item Amînarea evaluării lui \ccode{y} în expresia booleană \ccode{x || y}, cu speranța că \ccode{x} va fi evaluat ca adevărat, iar \ccode{y} va deveni irelevant.

  \item Inițializarea unei componente costisitoare dintr-un program doar cînd devine necesară (o conexiune la baza de date, o zonă a hărții dintr-un joc).
\end{itemize}

Așadar, definim un al doilea vector numit \ccode{lazy} și executăm \ccode{lazy[x] += 100} pe pozițiile 9, 5, 6 și 28.

Motivul pentru care treaba se complică este următorul. Dacă acum primim o interogare de sumă pe intervalul [25,29]? Nu putem să însumăm, ca de obicei, pozițiile 25, 13 și 14, căci pierdem din vedere că unele dintre noduri au (cîte) +100. Sigur, putem lua asta în calcul, dar trebuie să clarificăm operațiile, altfel efortul poate deveni $\bigoh(n)$.

În primul rînd, introducem două funcții noi (le puteți include în alte funcții, dar pentru claritate le puteți declara de sine stătătoare):

\begin{itemize}
  \item \ccode{push()}, care propagă informația \textit{lazy} de la un nod la fiii săi;
  \item \ccode{pull()}, care combină în părinte informația din cei doi fii după o actualizare.
\end{itemize}

\begin{minted}{c}
void push(int node, int size) {
  s[node] += lazy[node] * size;
  lazy[2 * node] += lazy[node];
  lazy[2 * node + 1] += lazy[node];
  lazy[node] = 0;
}

void pull(int node, int size) {
  s[node] =
    s[2 * node] + lazy[2 * node] * size / 2 +
    s[2 * node + 1] + lazy[2 * node + 1] * size / 2;
}
\end{minted}

Pentru problema dată (dar nu pentru toate problemele), codul are nevoie să știe numărul de frunze subîntinse (\ccode{size}).

Să presupunem acum că dorim să calculăm suma intervalului [2, 12] și că este posibil să avem niște sume \textit{lazy} în multe alte noduri. Știm că codul descompune interogările în intervale mai scurte și nu urcă mai sus de acestea. Dacă există valori \textit{lazy} mai sus (să zicem în rădăcină), codul nu va afla de ele. De aceea, în pregătirea interogării, trebuie să vizităm toți strămoșii intervalului și să propagăm în jos (\textit{push}) informația \textit{lazy}. Dar, dacă ne gîndim, lista completă a acestor strămoși constă doar din strămoșii capetelor de interval! Pentru intervalul [18,28], este nevoie să propagăm în jos informația \textit{lazy} din strămoșii lui 18 (adică 1, 2, 4 și 9) și ai lui 28 (adică 1, 3, 7 și 14).

Dacă apelăm push din acești strămoși, de sus în jos, avem garanția că informația pe intervalele dorite este la zi. Vă rămîne vouă ca experiment de gîndire să demonstrați că, după operațiile \textit{push}, nu va mai exista informație \textit{lazy} în niciun strămoș al niciunei poziții din interogare.

Astfel obținem o funcție foarte similară cu cea din capitolul trecut

\begin{minted}{c}
void push_path(int node, int size) {
  if (node) {
    push_path(node / 2, size * 2);
    push(node, size);
  }
}

long long query(int l, int r) {
  long long sum = 0;
  int size = 1;

  l += n;
  r += n;
  push_path(l / 2, 2); // pornim din părinte
  push_path(r / 2, 2);

  while (l <= r)  {
    if (l & 1) {
      sum += s[l] + lazy[l] * size;
      l++;
    }
    l >>= 1;

    if (!(r & 1)) {
      sum += s[r] + lazy[r] * size;
      r--;
    }
    r >>= 1;
    size <<= 1;
  }

  return sum;
}
\end{minted}

Dacă viteza este crucială, putem scrie și o funcție \ccode{push_path} cu circa 10\% mai rapidă, iterativă, folosind operații pe biți. Să considerăm nodul $22=10110_{(2)}$. Strămoșii lui sînt 1, 2, 5 și 11 care au respectiv reprezentările binare 1, 10, 101 și 1011, care sînt fix prefixele lui 10110! Deci îl vom deplasa pe 10110 la dreapta cu 4, 3, 2 și respectiv 1 bit pentru a-i obține strămoșii.

\begin{minted}{c}
void push_path(int node) {
  int bits = 31 - __builtin_clz(n);
  for (int b = bits, size = n; b; b--, size >>= 1) {
    int x = node >> b;
    push(x, size);
  }
}

// Acum primul apel este chiar din frunză:
...
push_path(l);
push_path(r);
...
\end{minted}

Actualizările sînt foarte similare. Apelăm \ccode{pull()} după terminarea actualizărilor, deoarece trebuie să lăsăm arborele într-o stare coerentă și trebuie să preluăm orice modificare de la fii.

\begin{minted}{c}
void pull_path(int node) {
  for (int x = node / 2, size = 2; x; x /= 2, size <<= 1) {
    pull(x, size);
  }
}

void update(int l, int r, int delta) {
  l += n;
  r += n;
  int orig_l = l, orig_r = r;
  push_path(l / 2, 2);
  push_path(r / 2, 2);

  while (l <= r)  {
    if (l & 1) {
      lazy[l++] += delta;
    }
    l >>= 1;

    if (!(r & 1)) {
      lazy[r--] += delta;
    }
    r >>= 1;
  }

  pull_path(orig_l);
  pull_path(orig_r);
}
\end{minted}

Iată și \href{https://codeforces.com/blog/entry/18051}{o altă implementare} care combină bucla \ccode{while} principală cu funcția \ccode{pull_path}.

Notă: În această implementare, valoarea \textit{lazy} se aplică întregului subarbore, inclusiv nodului însuși. În implementarea de pe \href{https://cp-algorithms.com/data_structures/segment_tree.html#adding-on-segments-querying-for-maximum}{CP Algorithms}, valoarea \textit{lazy} se aplică subarborelui fără nodul însuși. Oricare dintre formulări este acceptabilă, cîtă vreme o folosiți consecvent.

Notă: În practică, cîmpurile \ccode{lazy} și \ccode{s} merită încapsulate într-un \ccode{struct}. Datorită localității acceselor la memorie, diferența de viteză este notabilă (circa 25\%). Aici le-am lăsat separate pentru concizie.

\section{Implementare recursivă (actualizări punctuale)}

Lecția trecută am spus că există și o implementare recursivă. Să o examinăm acum (mulți o știți deja).

\begin{minted}{c}
void update(int node, int pl, int pr, int pos, int delta) {
  if (pr - pl == 1) {
    s[node] += delta;
  } else {
    int mid = (pl + pr) >> 1;
    if (pos < mid) {
      update(2 * node, pl, mid, pos, delta);
    } else {
      update(2 * node + 1, mid, pr, pos, delta);
    }
    s[node] = s[2 * node] + s[2 * node + 1];
  }
}
\end{minted}

Metoda recursivă cară după ea 5 parametri:

\begin{itemize}
  \item \ccode{node}: nodul curent din arbore (aka poziția în vector);

  \item \ccode{pl}, \ccode{pr}: intervalul din vectorul inițial acoperit de \ccode{node}. Eu am optat pentru implementarea cu \ccode{pl} inclusiv și \ccode{pr} exclusiv. Dacă preferați intervale închise, este OK.

  \item \ccode{pos}, \ccode{delta}: poziția de modificat și valoarea de adăugat/scăzut.
\end{itemize}

Vedem că funcția coboară recursiv în fiul stîng sau fiul drept, după caz. Un exemplu de apel ar fi:

\begin{minted}{c}
update(1, 0, n, some_pos, some_val);
\end{minted}

Mai interesant, iată și implementarea funcției de interogare (sumă pe interval):

\begin{minted}{c}
long long query(int node, int pl, int pr, int l, int r) {
  if (l >= r) {
    return 0;
  } else if ((l == pl) && (r == pr)) {
    return s[node];
  } else {
    int mid = (pl + pr) >> 1;

    return
      query(node * 2, pl, mid, l, min(r, mid)) +
      query(node * 2 + 1, mid, pr, max(l, mid), r);
  }
}
\end{minted}

Regăsim trei din aceiași parametri, \ccode{node}, \ccode{pl} și \ccode{pr}. În plus,

\begin{itemize}
  \item \ccode{l}, \ccode{r}: Intervalul [închis, deschis) pe care dorim să calculăm suma.
\end{itemize}

Funcția se reapelează pe cei doi fii, restrîngînd corespunzător intervalul $[l, r)$. Iată o imagine care arată arborele de apeluri pentru calculul sumei pe intervalul [3, 10):

\import{./figures}{segment-tree-rec-query.tex}

Am ilustrat cu albastru nodurile care au nevoie să-și apeleze descendenții, cu verde nodurile selectate integral, iar cu roșu nodurile eliminate.

Complexitatea rămîne $\mathcal{O}(\log n)$, deși funcția se reapelează pentru ambii fii. De ce?

Discutăm implementarea recursivă pentru că ea plutește prin supa culturală și vreau să puteți citi cod scris astfel. Dar ea este un exemplu de dopaj, de implementare repetată \textit{mot à mot} indiferent de nevoile problemei. Implementarea recursivă este de 2-3 ori mai lentă decît cea iterativă pentru actualizări punctuale. Presupun că există două motive:

Implementarea recursivă cară după ea 5-6 parametri la fiecare apel, care trebuie copiați, puși/scoși de pe stivă etc. Implementarea iterativă folosește doar 3 variabile.

Implementarea recursivă este nevoită să pornească din rădăcină, să coboare pînă la frunze, apoi să revină din recursivitate. Implementarea iterativă se oprește imediat ce termină de descompus intervalul $[l,r]$.

La varianta cu propagare \textit{lazy} diferența de timp aproape dispare, pentru că ambele implementări trebuie să urce pînă la rădăcină.

Nu vă năpustiți la implementarea recursivă dacă nu este nevoie. Rezistați tentației de a fi leneși, de a învăța o singură structură de date, pe care să o pictați indiferent de situație! Trebuie să aspirați la mai mult de atît, dacă este să vă meritați locul în lot.

\section{Implementare recursivă (actualizări pe interval)}

În această implementare, observăm cum:

\begin{itemize}
  \item apelăm \ccode{push} înainte de reapelarea recursivă, pentru a-i garanta fiecărui nod că deasupra sa nu mai există informații lazy;

  \item apelăm \ccode{pull} după revenirea din recursivitate, ca să lăsăm arborele într-o stare coerentă.
\end{itemize}

\begin{minted}{c}
long long query(int node, int pl, int pr, int l, int r) {
  if (l >= r) {
    return 0;
  } else if ((l == pl) && (r == pr)) {
    return s[node] + lazy[node] * (r - l);
  } else {
    push(node, pr - pl);
    int mid = (pl + pr) >> 1;
    return
      query(node * 2, pl, mid, l, min(r, mid)) +
      query(node * 2 + 1, mid, pr, max(l, mid), r);
  }
}

void update(int node, int pl, int pr, int l, int r, int delta) {
  if (l >= r) {
    return;
  } else if ((l == pl) && (r == pr)) {
    lazy[node] += delta;
  } else {

    push(node, pr - pl);
    int mid = (pl + pr) >> 1, child_size = (pr - pl) >> 1;
    update(2 * node, pl, mid, l, min(r, mid), delta);
    update(2 * node + 1, mid, pr, max(l, mid), r, delta);
    pull(node, child_size);

  }
}

void process_ops() {
  for (int i = 0; i < num_queries; i++) {
    if (q[i].t == OP_UPDATE) {
      update(1, 0, n, q[i].l - 1, q[i].r, q[i].val);
    } else {
      answer[num_answers++] = query(1, 0, n, q[i].l - 1, q[i].r);
    }
  }
}
\end{minted}
