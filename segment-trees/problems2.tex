\section{Probleme}

\subsection{Problema Polynomial Queries (CSES)}
\label{problem:polynomial-queries}

\href{https://cses.fi/problemset/task/1736}{enunț}
$\bullet$
\hyperref[code:polynomial-queries]{surse}

Problema seamănă mult cu cea discutată la teorie, dar pe intervale nu mai adăugăm constante, ci progresii aritmetice. Așadar, pare natural să reținem exact această informație \textit{lazy}: în fiecare nod reținem că în fiecare frunză acoperită de acel nod trebuie să adăugăm cîte un termen al unei progresii cu un anumit prim element și pasul (deocamdată) 1. De exemplu, dacă în figura \ref{fig:segment-tree-range-add} facem o actualizare pe intervalul [18, 28], atunci în nodul 5 notăm progresia cu primul termen 3 și pasul 1. Informația \textit{lazy} este o pereche $\langle 3,1 \rangle$.

Trebuie tratate atent diversele cazuri care iau naștere. Dacă două progresii acoperă același interval, vor lua naștere progresii cu pas mai mare decît 1. Să luăm un exemplu:

\begin{itemize}
  \item Progresia cu primul termen 5 și pasul 3, așadar 5, 8, 11, 14, ...
  \item Progresia cu primul termen 2 și pasul 7, așadar 2, 9, 16, 23, ...
  \item După însumare dorim să avem termenii 7, 17, 27, 37, ...
  \item Rezultă că suma este și ea o progresie cu primul termen 7 și pasul 10. Cu alte cuvinte, informațiile \textit{lazy} se pot compune ușor: $\langle 5,3 \rangle + \langle 2,7 \rangle = \langle 7,10 \rangle$.
\end{itemize}

La propagarea în jos a informației \textit{lazy}, în cei doi fii vom adăuga progresii cu același pas. În fiul drept, primul termen trebuie calculat, dar este ușor. Dacă într-un nod care acoperă 16 elemente avem o progresie cu primul element 3 și pasul 5, atunci fiul drept va începe cu al nouălea termen al progresiei:

$$3 + 5 \cdot (16/2) = 43$$

La operațiile de adăugare, pe toate intervalele din descompunere vom aduna progresii cu pasul 1, dar primul element diferă pentru fiecare interval (la fel, nu este greu de calculat).

Contractul pe care l-am ales pentru implementarea iterativă este:

\begin{itemize}
  \item \ccode{first} și \ccode{step} înseamnă că pe nodurile din intervalul acoperit trebuie adăugate valorile \ccode{first}, \ccode{first + step}, \ccode{first + 2 * step}, ...
  \item Valoarea \ccode{s} din fiecare nod \textbf{nu} include și suma progresiei dată de \ccode{<first, step>} din acel nod.
\end{itemize}

\subsection{Problema Nezzar and Binary String (Codeforces)}
\label{problem:nezzar-and-binary-string}

\href{https://codeforces.com/contest/1478/problem/E}{enunț}
$\bullet$
\hyperref[code:nezzar-and-binary-string]{sursă}

Problema este relativ directă odată ce „ne prindem” că trebuie să procesăm operațiile în ordine inversă. Știm șirul final $f$ și fie $[l, r]$ ultimul interval inspectat de Nanako. În momentul inspecției, șirul curent trebuia să fie identic cu $f$ pe pozițiile $[1, l) \cup (r,n]$, căci pe acelea nu le putem modifica. Pe pozițiile $[l, r]$ trebuiau să fie doar biți 0 sau doar biți 1. Care dintre ele? Știm că la ultima modificare am modificat strict mai puțin de jumătate din biți. Să notăm cu $z$ numărul de zerouri și cu $u$ numărul de unu de pe pozițiile $[l,r]$ din $f$. Iau naștere trei cazuri:

\begin{enumerate}
  \item Dacă $z > u$, înseamnă că la pasul anterior $[l,r]$ conținea doar 0.
  \item Dacă $z < u$, înseamnă că la pasul anterior $[l,r]$ conținea doar 1.
  \item Dacă $z = u$, problema nu are soluție, căci nu putem opera modificarea necesară.
\end{enumerate}

Astfel, toate operațiile sînt forțate, mergînd înapoi în timp. Răspunsul este \texttt{YES} doar dacă putem procesa toate operațiile, iar la final ajungem la șirul $s$.

Rezultă că, pentru a efectua efectiv operațiile, avem nevoie de un arbore de segmente cu valori de 0 și 1 în frunze, cu funcții de sumă pe interval (pentru a stabili majoritatea) și de atribuire pe interval (pentru a face \textit{undo} la o operație).

\subsection{Problema Simple (infO(1)Cup 2019)}
\label{problem:simple}

\href{https://kilonova.ro/problems/3424/}{enunț}
$\bullet$
\hyperref[code:simple]{sursă}

Să aplicăm regula menționată ca să proiectăm un arbore de intervale pentru această problemă.

\begin{itemize}
  \item În cîmpul \textit{lazy} vom stoca valorile primite la update, așadar cantitățile de adăugat pe tot subarborele.
  \item În cîmpurile propriu-zise vom stoca valorile necesare pentru interogări, așadar minimul par pe interval și maximul impar pe interval.
  \item Ce altceva ne mai trebuie ca să putem menține informația la actualizări? Să observăm că o cantitate \textit{lazy} impară schimă paritatea valorilor pe întregul interval. Noul minim par este fostul minim impar, plus  cantitatea \textit{lazy}. De acea, vom introduce încă două cantități: maximul par și minimul impar.
\end{itemize}

Ca de obicei, este important ca la implementare să alegem dacă valoarea \textit{lazy} este deja inclusă în nodul curent și mai trebuie aplicată doar la subarbore sau dacă ea trebuie aplicată inclusiv nodului curent. Eu am ales prima variantă. Ambele sînt bune, cîtă vreme codul respectă alegerea făcută.

Pe unele intervale nu vor exista valori pare sau impare. Dacă facem cazuri speciale pentru toate acele situații, vom avea undeva între 5 și 10 \ccode{if}-uri de presărat prin cod. O abordare mai simplă este să notăm în acele noduri $+\infty$ pentru a arăta că nu există minime și $-\infty$ pentru a arăta că nu există maxime. Apoi lăsăm aceste valori să se combine fără să mai tratăm cazuri particulare. La final, știm că orice valori definite vor fi între $1$ și $2 \cdot 20^9 + 2 \cdot 20^5 \cdot 2 \cdot 20^9$, conform limitelor din enunț. Valorile din afara acestui interval le interpretăm ca fiind nedefinite.

În cod am încercat să separ funcțiile specifice nodului de funcțiile specifice arborelui.

\subsection{Problema Balama (Baraj ONI 2024)}
\label{problem:balama}

\href{https://kilonova.ro/problems/2666}{enunț}
$\bullet$
\hyperref[code:balama]{surse}

Vă veți întîlni des cu probleme unde soluția devine simplă dacă analizăm informațiile în altă ordine. În cazul de față, în loc să luăm în calcul liniile (care sînt subsecvențe ordonate), să analizăm coloanele.

Care va fi răspunsul pe ultima coloană? Desigur, va fi maximul din vector. Mult mai interesantă este întrebarea: care va fi răspunsul pe penultima coloană? Va fi cel mai mare element care este vreodată (în cel puțin o fereastră) \textbf{al doilea maxim}.

Exemplu: fie maximul global $1.000$ și fie al doilea maxim global $999$. Dacă $999$ stă foarte departe de $1.000$, la distanță de cel puțin $k$, atunci $999$ va fi maxim în toate ferestrele de lățime $k$ care îl conțin. Cu alte cuvinte, $999$ se va regăsi doar pe ultima coloană în matricea $B$ (și va fi mascat de $1.000$). Să spunem că următoarele valori din șir, în ordine descrescătoare, sînt $998$, $997$ și $996$ și toate se află la distanță mare unele de altele. Niciunul dintre ele nu va apărea pe penultima coloană în $B$.

În schimb, să spunem că următoarea valoare ca mărime, $995$, se află aproape (la distanță $< k$) de o valoare anterioară, cum ar fi $997$. Atunci există o fereastră în care $995$ și $997$ coexistă, deci $995$ va fi al doilea maxim din acea fereastră și va apărea pe penultima coloană în $B$. Cum alte valori mai mari nu au această proprietate, $995$ este răspunsul pe penultima poziție a soluției.

(Amănunt esențial \emoji{winking-face}: $995$ nu poate fi și al treilea maxim. Dacă exista o fereastră de lățime $k$ care îl cuprindea pe $995$ și alte două valori anterioare, atunci înainte să ajungem la $995$ una dintre acele două valori anterioare ar fi fost al doilea maxim).

Astfel, putem considera elemente în ordine descrescătoare și, pentru fiecare element $x$ ne întrebăm: cîte elemente văzute anterior conține fiecare dintre ferestrele care îl conțin pe $x$ (cel mult $k$ la număr)? Dacă o astfel de fereastră are $e$ elemente, și dacă a $e+1$-a valoare din soluție (numărînd de la dreapta) este încă necunoscută, atunci pune $x$ pe poziția $e+1$ a soluției.

Exemplu: Dacă considerăm elementul $900$ și constatăm că într-una din ferestrele care îl conțin pe $900$ existau deja alte $5$ valori, atunci în acea fereastră $900$ este al $6$-lea element. Dacă a $6$-a poziție din soluție este încă neocupată, scriem $900$ acolo, acesta fiind maximul posibil.

Implementarea sună fioros, dar nu este! În realitate avem nevoie de o structură cu două operații:

\begin{itemize}
  \item Incrementează pozițiile de la $st$ la $dr$, pentru a arăta că în ferestrele de la $[st, st+k-1]$ și pînă la $[dr, dr + k - 1]$ avem cîte un element în plus.
  \item Află maximul de pe pozițiile de la $st$ la $dr$.
\end{itemize}

Operația a doua ne este suficientă deoarece ferestrele nu vor ajunge brusc la 6 elemente. Vor apărea mai întîi ferestre cu 1, 2, 3, 4, 5 elemente. Cu alte cuvinte, soluția se completează de la dreapta spre stînga.

Vom implementa un AINT de maxime în care informația \textit{lazy} din fiecare nod este valoarea de adăugat pe fiecare poziție din intervalul acoperit.
