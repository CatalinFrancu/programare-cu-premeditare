\section{Probleme}

\subsection{Problema Gold Transfer (Codeforces)}
\label{problem:gold-transfer}

\href{https://codeforces.com/contest/1535/problem/E}{enunț}
$\bullet$
\hyperref[code:gold-transfer]{sursă}

Atenție mare la garanția din enunț: $c_i > c_{p_i}$. Cu alte cuvinte, pentru orice operație de cumpărare trebuie să pornim din rădăcină spre nodul $u$ și să cumpărăm orice cantități disponibile, pînă satisfacem cererea.

Cu timpul, nodurile de la rădăcină se vor goli, deci pare o idee bună să găsim rapid cel mai de jos strămoș care încă are aur disponibil. Dacă reușim să-l găsim în $\bigoh(f(n))$ (intenția problemei fiind $f(n) = \log n$), atunci complexitatea globală va fi $\bigoh(q \cdot f(n) + n)$. De ce? Din acel strămoș putem parcurge lanțul în jos pas cu pas, cumpărînd tot aurul disponibil, pînă satisfacem cererea. Fiecare nod poate fi golit cel mult o dată, deci efortul total pentru parcurgerea lanțurilor este $\bigoh(n)$.

\subsubsection*{Detalii de implementare}

Arborele este dinamic, deci nu putem construi o liniarizare.

Am ales să accelerez urcarea în arbore cu \textit{binary lifting} cu doi pointeri, dar orice altă metodă este acceptabilă.

Odată ce golim un nod, avem nevoie să coborîm în fiul său care duce spre nodul original. Am ales să  fac acest lucru cu o stivă, dar soluția este cam lentă (2x față de altele).

Întrucît nu avem de ce să parcurgem toți fiii unui nod, nu avem nevoie de liste de adiacență, ci doar de pointeri la părinte.

\subsection{Problema A and B and Lecture Rooms (Codeforces)}
\label{problem:a-and-b-and-lecture-rooms}

\href{https://codeforces.com/contest/519/problem/E}{enunț}
$\bullet$
\hyperref[code:a-and-b-and-lecture-rooms]{sursă}

Problema cere să răspundem la $m$ întrebări de forma: Date fiind nodurile $u$ și $v$ (posibil egale), cîte noduri din arbore se află la distanță egală de $u$ și de $v$?

Pentru soluția teoretică, următoarea vizualizare este utilă. Să „atîrnăm” arborele de nodurile $u$ și $v$, ca pe o ghirlandă bine întinsă.  Atunci calea cea mai scurtă $u-v$ va fi orizontală, iar de lanțurile de pe cale vor atîrna restul subarborilor.

Dacă distanța $u-v$ este impară, atunci răspunsul este 0. Dacă distanța este pară, atunci fie $w$ nodul de la jumătatea distanței. Nodurile aflate la distanță egală de $u$ și de $v$ vor fi $w$ și orice nod din subarborii care atîrnă din $w$.

În practică vom alege o rădăcină și vom precalcula informații pentru LCA. Apoi, eu am redus problema la următoarele cazuri (poate se poate și mai simplu):

\begin{enumerate}
  \item Dacă distanța $u-v$ este impară, răspunsul este 0.

  \item Dacă $u=v$, răspunsul este $n$.

  \item Dacă $u$ și $v$ au adîncimi egale, atunci răspunsul este: mărimea subarborelui lui $LCA(u,v)$ minus mărimile subarborilor fiilor lui $LCA(u,v)$ care pornesc către $u$, respectiv către $v$.

  \item Altfel, să presupunem că $u$ are adîncime mai mare decît $v$. Atunci nodul $w$ este undeva mai jos de LCA, mergînd către $u$. Răspunsul este: mărimea subarborelui lui $w$ minus mărimea subarborelui fiului lui $w$ care pornește către $u$.
\end{enumerate}

\subsubsection*{Detalii de implementare}

Punctul (4) presupune să calculăm al $k$-lea strămoș. De exemplu, dacă adîncimile sînt $d[u]=50$, $d[v]=20$ și $d[LCA]=10$, atunci distanța $u-v$ este $40+10=50$, jumătatea distanței este $25$, iar nodul $w$ este al $25$-lea strămoș al lui $u$.

De aceea, metoda lui Tarjan nu ne prea ajută, ci avem nevoie de o metodă cu \textit{jump pointers}.

La punctele (3) și (4), pentru a afla „fiul lui $w$ care pornește către $u$”, putem refolosi informațiile pentru al $k$-lea strămoș. Dacă notăm cu $k$ diferența pe înălțime între $w$ și $u$, atunci răspunsul este al $k-1$-lea strămoș al lui $u$.

La arbori, multe informații sînt redundante și putem stoca doar o submulțime. Exemple:

\begin{itemize}
  \item Mărimea subarborelui lui $u$ nu trebuie stocată implicit dacă cunoaștem timpii DFS, ci poate fi calculată ca $t_o[u]-t_i[u]+1$.

  \item Paritatea distanței poate fi calculată fără a calcula distanța efectivă. Calculăm doar suma adîncimilor.

  \item Al $k$-lea strămoș al unui nod poate fi găsit ca strămoșul de la adîncimea $d[u]-k$ al unui nod, dacă cunoaștem adîncimile nodurilor.
\end{itemize}
