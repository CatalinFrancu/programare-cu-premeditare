\part{Structuri de date pe vectori}

Următoarele capitole tratează structuri de date care pot procesa anumite operații pe vectori în timp mai bun decît $\bigoh(N)$. Ocazional aceste structuri se aplică și matricilor.

Subiectele de ONI / baraj ONI / lot din anii trecuți abundă în probleme rezolvabile cu astfel de structuri:

\begin{itemize}
  \item \href{https://kilonova.ro/problems/141}{3dist} (baraj ONI 2022)
  \item \href{https://kilonova.ro/problems/2801}{6 de Pentagrame} (lot 2024)
  \item \href{https://kilonova.ro/problems/3734}{Babel} (baraj ONI 2025)
  \item \href{https://kilonova.ro/problems/2666}{Balama} (baraj ONI 2024)
  \item \href{https://kilonova.ro/problems/63}{Bisortare} (ONI 2021)
  \item \href{https://kilonova.ro/problems/3806}{Circuit} (lot 2025)
  \item \href{https://kilonova.ro/problems/76}{Emacs} (baraj ONI 2021)
  \item \href{https://kilonova.ro/problems/210}{Erinaceida} (lot 2022)
  \item \href{https://kilonova.ro/problems/210}{Guguștiuc} (baraj ONI 2022)
  \item \href{https://kilonova.ro/problems/565}{Împiedicat} (baraj ONI 2023)
  \item \href{https://kilonova.ro/problems/137}{Lupușor} (ONI 2022)
  \item \href{https://kilonova.ro/problems/3790}{Medwalk} (lot 2025)
  \item \href{https://kilonova.ro/problems/2669}{Perm} (baraj ONI 2024)
  \item \href{https://kilonova.ro/problems/142}{Piezișă} (baraj ONI 2022)
  \item \href{https://kilonova.ro/problems/2843}{Subiectul III} (lot 2024)
  \item \href{https://kilonova.ro/problems/556}{Șirbun} (baraj ONI 2023)
  \item \href{https://kilonova.ro/problems/3790}{Trapez} (lot 2025)
\end{itemize}

Pare o idee bună să le învățăm și să le stăpînim bine. \emoji{slightly-smiling-face} Concret, vom studia trei structuri:

\begin{enumerate}
  \item arbori de intervale;
  \item arbori indexați binar;
  \item descompunere în radical.
\end{enumerate}

Vom exemplifica structurile și vom face benchmarks pe două probleme didactice. Apoi vom vedea, prin probleme, cum putem extinde aceleași structuri pentru nevoi mai complicate.

\textbf{Varianta 1 (actualizări punctuale, interogări pe interval)}: Se dă un vector de $N$ elemente întregi și $Q$ operații de două tipuri:

\begin{enumerate}
  \item $\langle 1, x, val \rangle$: Adaugă $val$ pe poziția $x$ a vectorului.
  \item $\langle 2, x, y \rangle$: Calculează suma pozițiilor de la $x$ la $y$ inclusiv.
\end{enumerate}

\textbf{Varianta 2 (actualizări pe interval, interogări pe interval)}: Similar, dar operația 1 este pe interval:

\begin{enumerate}
  \item $\langle 1, x, y, val \rangle$: Adaugă $val$ pe pozițiile de la $x$ la $y$ inclusiv.
  \item $\langle 2, x, y \rangle$: Calculează suma pozițiilor de la $x$ la $y$ inclusiv.
\end{enumerate}

Vom menționa ocazional și \textbf{Varianta 3 (actualizări pe interval, interogări punctuale)}:

\begin{enumerate}
  \item $\langle 1, x, y, val \rangle$: Adaugă $val$ pe pozițiile de la $x$ la $y$ inclusiv.
  \item $\langle 2, x \rangle$: Returnează valoarea poziției $x$.
\end{enumerate}

Toate implementările mele sînt disponibile \href{https://github.com/CatalinFrancu/nerdvana/tree/main/query-update}{pe GitHub}.

\begin{refsection}
  \import{./segment-trees}{segment-trees.tex}
  \import{./segment-trees}{problems1.tex}

  \nocite{cpa-segtree}
  \nocite{csa-segtree}
  \printbibliography
\end{refsection}

\begin{refsection}
  \import{./segment-trees}{segment-trees-lazy.tex}
\end{refsection}
