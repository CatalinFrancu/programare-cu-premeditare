\section{Probleme}

\subsection{Problema The Permutation Game Again (SPOJ)}
\label{problem:the-permutation-game-again}

\href{https://www.spoj.com/problems/TPGA/}{enunț}
$\bullet$
\hyperref[code:the-permutation-game-again]{sursă}

Problema ne cere să aflăm \textbf{rangul} unei permutări (engl. \textit{rank}). Acesta este numărul de ordine al permutării în lista ordonată lexicografic a tuturor permutărilor mulțimii $\{1, 2, \dots, n\}$.

Echivalent, trebuie să răspundem eficient la întrebarea: cîte permutări vin înaintea celei date în lista permutărilor?

Să considerăm un exemplu. Dacă primul element al permutării este 9, atunci toate permutările care încep cu $1, 2, \dots, 8$ o vor preceda în listă. Există $8(n-1)!$ astfel de permutări.

Similar, dacă al doilea element este 3, atunci toate permutările care încep cu $91$ sau $92$ o vor preceda în listă. Există $2(n-2)!$ astfel de permutări.

Dar dacă al treilea element este 7? Acum trebuie să ținem cont de faptul că pe 3 l-am văzut deja. Trebuie să socotim permutările care încep cu $931, 932, 934, 935, 936$. Există $5(n-3)!$ astfel de permutări.

Cu alte cuvinte, trebuie să răspundem eficient la întrebarea: cîte elemente mai mici decît cel curent am văzut în prefixul dinaintea elementului curent? Putem gestiona această informație cu un AIB de 0 și 1. Cînd procesăm un element de valoare $x$, adunăm 1 pe poziția $x$ în AIB. Astfel, suma parțială din AIB pe o poziție $y$ ne va arăta cîte elemente mai mici decît $y$ am procesat pînă în prezent.

Pentru un plus de eficiență, sursa nu reține întreaga permutare, ci doar citește cîte un element, îl ia în calcul la rang, îl bifează în AIB, apoi îl aruncă.

\subsection{Problema Multiset (Codeforces)}
\label{problem:multiset}

\href{https://codeforces.com/contest/1354/problem/D}{enunț}
$\bullet$
\hyperref[code:multiset]{sursă}

„Aproape” putem rezolva problema cu un singur vector de frecvențe. Dar avem nevoie să găsim eficient al $k$-lea element ca să-l putem șterge. Un vector de frecvențe ne dă inserări în $\bigoh(1)$, dar ștergeri în $\bigoh(n)$.

De aceea, înlocuim vectorul cu un AIB în care pe poziția $x$ notăm frecvența lui $x$ în multiset. Astfel putem căuta al $k$-lea element reformulînd definiția: al $k$-lea element este poziția $p$ pe care suma parțială atinge sau depășește valoarea $k$.
