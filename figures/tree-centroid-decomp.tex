\begin{figure}[H]
  \centering

  \begin{tikzpicture}[
    l1/.style={ text=white, fill=red },
    l2/.style={ text=white, fill=orange },
    l3/.style={ text=black, fill=yellow },
    l4/.style={ text=white, fill=Green },
    l5/.style={ text=white, fill=blue },
    perim2/.style={ orange, thick, densely dotted },
    perim3/.style={ yellow!80!black, thick, densely dotted },
    ]
    \graph [nodes={draw, circle, minimum size=22pt}, no placement] {
      1[l1, at={(0,0)}] -- {
        2[l2, at={(0,-3)}],
        3[l4, at={(-3,0)}] -- {
          4[l3, at={(-3,2)}],
          5[l2, at={(-5,0)}] -- 6[l3, at={(-5,-2)}] -- {
            7[l4, at={(-3,-2)}],
            8[l4, at={(-5,-4)}],
          },
        },
        9[l5, at={(3,0)}] -- {
          10[l4, at={(3,-2)}],
          11[l3, at={(5,0)}] -- {
            12[l4, at={(7,0)}],
            13[l4, at={(5,-2)}] -- 14[l2, at={(5,-4)}] -- {
              15[l3, at={(7,-4)}],
              16[l3, at={(5,-6)}] -- {
                17[l4, at={(3,-6)}],
                18[l4, at={(7,-6)}],
              },
            },
          },
        },
        19[l3, at={(0,3)}] -- {
          20[l4, at={(-1.2,1.8)}],
          21[l4, at={(-1.2,4.2)}],
          22[l4, at={(0,5)}],
          23[l2, at={(2,3)}] -- 24[l5, at={(4,3)}] -- {
            25[l4, at={(6,3)}],
            26[l3, at={(4,5)}] -- 27[l4, at={(6,5)}],
          },
        },
      };
    };

    \draw[perim2] (-0.7,-3.7) rectangle (0.7,-2.3);
    \draw[perim2] (-5.7,-4.7) rectangle (-2.3,2.7);
    \draw[perim2] (2.3,0.7) rectangle (7.7,-6.7);
    \draw[perim2] (-1.9,1.1) rectangle (6.7,5.7);

    \draw[perim3] (-3.5,-0.5) rectangle (-2.5,2.5);
    \draw[perim3] (-5.5,-4.5) rectangle (-2.5,-1.5);
    \draw[perim3] (2.5,-2.5) rectangle (7.5,0.5);
    \draw[perim3] (6.5,-4.5) rectangle (7.5,-3.5);
    \draw[perim3] (2.5,-6.5) rectangle (7.5,-5.5);
    \draw[perim3] (-1.7,1.3) rectangle (0.5,5.5);
    \draw[perim3] (3.5,2.5) rectangle (6.5,5.5);
  \end{tikzpicture}

  \caption{Un arbore descompus în centroizi. Nodul 1 (roșu) este centroid de nivelul 1. Pentru subarborii rezultați prin eliminarea lui 1, nodurile portocalii sînt centroizi de nivelul 2. Nodurile galbene, verzi și albastre sînt centroizi de nivelurile 3, 4 și respectiv 5.}
  \label{fig:tree-centroid-decomp}
\end{figure}
