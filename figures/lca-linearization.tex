\begin{figure}[H]
  \centering

  \begin{tikzpicture}[
    scale=0.7,
    every node/.style={scale=0.7},
    endpoint/.style={fill=red!20},
    spanned/.style={fill=blue!20},
    ]
    \begin{scope}[
      every node/.style={
        circle,
        draw,
        inner sep=2pt,
        minimum size=20pt,
      },
      level distance=20mm,
      level 1/.style={sibling distance=70mm},
      level 2/.style={sibling distance=45mm},
      level 3/.style={sibling distance=20mm}
      ]
      \node {1}
      child {node {7}
        child {node {4}}
        child {node {9}}
      }
      child {node[spanned] {5}
        child {node[spanned] {11}
          child {node[endpoint] {2}}
          child {node[spanned] {14}}
          child {node[spanned] {6}}
        }
        child {node[spanned] {3}
          child {node[endpoint] {13}
            child {node {10}}
          }
          child {node {8}}
        }
      }
      child {node {12}};
    \end{scope}

    \begin{scope}[shift={(-10,-9.5)}]
      % First row: indices
      \matrix[right] (index) at (0, 0) [
      matrix of nodes,
      nodes={anchor=center, minimum size=20pt, color=black!50, font=\small},
      ] {
        1 & 2 & 3 & 4 & 5 & 6 & 7 & 8 & 9 & 10 & 11 & 12 & 13 & 14 & 15 & 16 &
        17 & 18 & 19 & 20 & 21 & 22 & 23 & 24 & 25 & 26 & 27\\
      };

      % Second row: Euler linearization
      \node[label] (label-abstract) at (-0.75, -0.75) {Euler};

      \matrix[right] (abstract) at (0, -0.75) [
      matrix of nodes,
      nodes={draw, anchor=center, minimum size=20pt},
      column sep=-\pgflinewidth,
      row sep=-\pgflinewidth
      ] {
        1 & 7 & 4 & 7 & 9 & 7 & 1 & 5 & 11 & \node[endpoint]{2}; &
        \node[spanned]{11}; & \node[spanned]{14}; & \node[spanned]{11}; &
        \node[spanned]{6}; & \node[spanned]{11}; & \node[spanned]{5}; &
        \node[spanned]{3}; & \node[endpoint]{13}; & 10 & 13 & 3 & 8 & 3 & 5 & 1
        & 12 & 1\\
      };

      % Third row: depths
      \node[label] (label-abstract) at (-0.75, -1.75) {adîncimi};

      \matrix[right] (abstract) at (0, -1.75) [
      matrix of nodes,
      nodes={draw, anchor=center, minimum size=20pt},
      column sep=-\pgflinewidth,
      row sep=-\pgflinewidth
      ] {
        0 & 1 & 2 & 1 & 2 & 1 & 0 & 1 & 2 & \node[endpoint]{3}; &
        \node[spanned]{2}; & \node[spanned]{3}; & \node[spanned]{2}; &
        \node[spanned]{3}; & \node[spanned]{2}; & \node[spanned]{1}; &
        \node[spanned]{2}; & \node[endpoint]{3}; & 4 & 3 & 2 & 3 & 2 & 1 & 0 & 1
        & 0\\ };
    \end{scope}
  \end{tikzpicture}

  \caption{Aflarea LCA printr-o liniarizare Euler cu repetiție. LCA(2,13) se reduce la o interogare de minim pe intervalul [10,18]. Adîncimea minimă este 1, corespunzătoare nodului 5.}
  \label{fig:lca-linearization}
\end{figure}
