\begin{figure}[H]
  \centering

  \newcommand\halfopen[2]{\ensuremath{[#1,#2)}}

  \tikzset{box1/.style={draw, align=left, font=\tiny, minimum size=0.9cm}}
  \tikzset{box2/.style={draw, align=left, font=\scriptsize, minimum height=0.9cm, minimum width=1.9cm}}
  \tikzset{box3/.style={draw, minimum height=0.9cm, minimum width=3.9cm}}
  \tikzset{box4/.style={draw, minimum height=0.9cm, minimum width=7.9cm}}
  \tikzset{box5/.style={draw, minimum height=0.9cm, minimum width=15.9cm}}
  \tikzset{label/.style={color=black!50, font=\tiny}}

  \begin{tikzpicture}

    \node [box5] at ( 7.5, 4) (node1)  {
      \halfopen{0}{16} \ \ \ {\color{blue} \halfopen{3}{10}}
    };

    \node [box4] at ( 3.5, 3) (node2)  {
      \halfopen{0}{8} \ \ \ {\color{blue} \halfopen{3}{8}}
    };
    \node [box4] at (11.5, 3) (node3)  {
      \halfopen{8}{16} \ \ \ {\color{blue} \halfopen{8}{10}}
    };

    \node [box3] at ( 1.5, 2) (node4)  {
      \halfopen{0}{4} \ \ \ {\color{blue} \halfopen{3}{4}}
    };
    \node [box3, fill=Green!10] at ( 5.5, 2) (node5)  {
      \halfopen{4}{8} \ \ \ {\color{Green} \halfopen{4}{8}}
    };
    \node [box3] at ( 9.5, 2) (node6)  {
      \halfopen{8}{12} \ \ \ {\color{blue} \halfopen{8}{10}}
    };
    \node [box3, fill=red!10] at (13.5, 2) (node7)  {
      \halfopen{12}{16} \ \ \ {\color{red} \halfopen{12}{10}}
    };

    \node [box2, fill=red!10] at ( 0.5, 1) (node8)  {
      \halfopen{0}{2}\\{\color{red} \halfopen{3}{2}}
    };
    \node [box2] at ( 2.5, 1) (node9)  {
      \halfopen{2}{4}\\{\color{blue} \halfopen{3}{4}}
    };
    \node [box2] at ( 4.5, 1) (node10) {};
    \node [box2] at ( 6.5, 1) (node11) {};
    \node [box2, fill=Green!10] at ( 8.5, 1) (node12) {
      \halfopen{8}{10}\\{\color{Green} \halfopen{8}{10}}
    };
    \node [box2, fill=red!10] at (10.5, 1) (node13) {
      \halfopen{10}{12}\\{\color{red} \halfopen{10}{10}}
    };
    \node [box2] at (12.5, 1) (node14) {};
    \node [box2] at (14.5, 1) (node15) {};

    \node [box1] at ( 0, 0) (node16) {};
    \node [box1] at ( 1, 0) (node17) {};
    \node [box1, fill=red!10] at ( 2, 0) (node18) {
      \halfopen{2}{3}\\{\color{red} \halfopen{3}{3}}
    };
    \node [box1, fill=Green!10] at ( 3, 0) (node19) {
      \halfopen{3}{4}\\{\color{Green} \halfopen{3}{4}}
    };
    \node [box1] at ( 4, 0) (node20) {};
    \node [box1] at ( 5, 0) (node21) {};
    \node [box1] at ( 6, 0) (node22) {};
    \node [box1] at ( 7, 0) (node23) {};
    \node [box1] at ( 8, 0) (node24) {};
    \node [box1] at ( 9, 0) (node25) {};
    \node [box1] at (10, 0) (node26) {};
    \node [box1] at (11, 0) (node27) {};
    \node [box1] at (12, 0) (node28) {};
    \node [box1] at (13, 0) (node29) {};
    \node [box1] at (14, 0) (node30) {};
    \node [box1] at (15, 0) (node31) {};

    \foreach \i in {1,...,31} {
      \node [label, below right=-0.1 of node\i.north west] {\i};
    }

  \end{tikzpicture}

  \caption{Arborele de apeluri în funcția de actualizare recursivă pe intervalul $[3,10)$.}
  \label{fig:segment-tree-rec-query}
\end{figure}
