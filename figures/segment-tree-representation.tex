\begin{figure}[H]
  \centering

  \tikzset{box1/.style={draw, minimum size=0.9cm}}
  \tikzset{box2/.style={draw, minimum height=0.9cm, minimum width=1.9cm}}
  \tikzset{box3/.style={draw, minimum height=0.9cm, minimum width=3.9cm}}
  \tikzset{box4/.style={draw, minimum height=0.9cm, minimum width=7.9cm}}
  \tikzset{box5/.style={draw, minimum height=0.9cm, minimum width=15.9cm}}
  \tikzset{label/.style={color=black!50, font=\tiny}}

  \begin{tikzpicture}
    \node [box5] at ( 7.5, 4) (node1)  {95};

    \node [box4] at ( 3.5, 3) (node2)  {49};
    \node [box4] at (11.5, 3) (node3)  {46};

    \node [box3] at ( 1.5, 2) (node4)  {19};
    \node [box3] at ( 5.5, 2) (node5)  {30};
    \node [box3] at ( 9.5, 2) (node6)  {18};
    \node [box3] at (13.5, 2) (node7)  {28};

    \node [box2] at ( 0.5, 1) (node8)  { 8};
    \node [box2] at ( 2.5, 1) (node9)  {11};
    \node [box2] at ( 4.5, 1) (node10) {14};
    \node [box2] at ( 6.5, 1) (node11) {16};
    \node [box2] at ( 8.5, 1) (node12) {11};
    \node [box2] at (10.5, 1) (node13) { 7};
    \node [box2] at (12.5, 1) (node14) { 9};
    \node [box2] at (14.5, 1) (node15) {19};

    \node [box1] at ( 0, 0) (node16) { 3};
    \node [box1] at ( 1, 0) (node17) { 5};
    \node [box1] at ( 2, 0) (node18) {10};
    \node [box1] at ( 3, 0) (node19) { 1};
    \node [box1] at ( 4, 0) (node20) { 9};
    \node [box1] at ( 5, 0) (node21) { 5};
    \node [box1] at ( 6, 0) (node22) { 7};
    \node [box1] at ( 7, 0) (node23) { 9};
    \node [box1] at ( 8, 0) (node24) { 5};
    \node [box1] at ( 9, 0) (node25) { 6};
    \node [box1] at (10, 0) (node26) { 1};
    \node [box1] at (11, 0) (node27) { 6};
    \node [box1] at (12, 0) (node28) { 2};
    \node [box1] at (13, 0) (node29) { 7};
    \node [box1] at (14, 0) (node30) {10};
    \node [box1] at (15, 0) (node31) { 9};

    \foreach \i in {1,...,31} {
      % \node [label, above left=-0.3 and -0.4 of node\i] {\i};
      \node [label, below right=-0.1 of node\i.north west] {\i};
    }
  \end{tikzpicture}

  \caption{Un arbore de intervale cu 16 frunze și 15 noduri interne. Valorile din fiecare celulă reprezintă suma din frunzele subîntinse de acea celulă. Cu cifre mici este notat indicele fiecărei celule.}
  \label{fig:segment-tree-representation}
\end{figure}
