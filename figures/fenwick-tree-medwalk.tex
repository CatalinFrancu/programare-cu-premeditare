\newcommand\FenwickTree[2]{ % anchor, data rows
  \matrix[
  at=(#1),
  anchor=north,
  draw,
  matrix of nodes,
  ampersand replacement=\&,
  column sep=3pt,
  nodes={minimum height=2em, text width=1cm, align=center, anchor=base},
  ]{
    \node{pos}; \& \node{AIB}; \\
    \node{...}; \& \node{...}; \\
    #2
    \node{...}; \& \node{...}; \\
  };
}

\begin{figure}[H]
  \centering

  \begin{tikzpicture}[scale=0.63, every node/.style={scale=0.63}]
    % Contents of a slice of the array of minima.
    \matrix[
    anchor=west,
    align=left,
    nodes={minimum height=1.5em, align=center, anchor=center},
    header/.style={text width=5cm, align=left},
    vpos/.style={text width=1cm, font=\small, text=black!50},
    val/.style={draw, text width=1cm},
    ]{
      \node[header]{poziția}; & \node{...}; & \node[vpos]{100}; & \node[vpos]{101}; &
      \node[vpos]{102}; & \node[vpos]{103}; & \node[vpos]{104}; & \node[vpos]{105}; &
      \node[vpos]{106}; & \node[vpos]{107}; & \node[vpos]{108}; & \node[vpos]{109}; &
      \node[vpos]{110}; & \node[vpos]{111}; & \node[vpos]{112}; & \node[vpos]{113}; &
      \node[vpos]{114}; & \node{...}; \\

      \node[header]{minimul acum}; & \node{...}; & \node[val]{10}; & \node[val]{1}; &
      \node[val]{9}; & \node[val]{1}; & \node[val]{11}; & \node[val]{12}; &
      \node[val]{10}; & \node[val]{1}; & \node[val]{12}; & \node[val]{9}; &
      \node[val]{9}; & \node[val]{1}; & \node[val]{12}; & \node[val]{9}; &
      \node[val]{1}; & \node{...}; \\

      \node[header]{minimele la alte momente}; & \node{...}; & & & & \node{10, 11}; &
      & & & & & & & \node{11}; & & & & \node{...}; \\
    };

    % Contents of the Fenwick tree of columns (values).
    \matrix[
    anchor=west,
    align=left,
    yshift=-6em,
    nodes={minimum height=1.5em, text width=3cm, align=center, anchor=center},
    header/.style={text width=3cm, align=left},
    ] {
      \node[header]{valoarea}; & \node{...}; & \node{9}; & \node{10}; & \node{11}; &
      \node{12}; & \node{...}; \\

      \node[header]{valori subîntinse}; &
      \node{...}; &
      \node (val-9) {9}; &
      \node (val-9-10) {9-10}; &
      \node (val-11) {11}; &
      \node (val-9-12) {9-12}; &
      \node{...}; \\
    };

    % Fenwick tree for value 9.
    \FenwickTree{val-9.south}{
      \node[draw]{102}; \& \node[draw]{1}; \\
      \node[draw]{109}; \& \node[draw]{1}; \\
      \node[draw]{110}; \& \node[draw]{1}; \\
      \node[draw]{113}; \& \node[draw]{1}; \\
    }

    % Fenwick tree for values 9-10.
    \FenwickTree{val-9-10.south}{
      \node[draw]{100}; \& \node[draw]{1}; \\
      \node[draw]{102}; \& \node[draw]{1}; \\
      \node[draw]{103}; \& \node[draw]{0}; \\
      \node[draw]{106}; \& \node[draw]{1}; \\
      \node[draw]{109}; \& \node[draw]{1}; \\
      \node[draw]{110}; \& \node[draw]{1}; \\
      \node[draw]{113}; \& \node[draw]{1}; \\
    }

    % Fenwick tree for values 11.
    \FenwickTree{val-11.south}{
      \node[draw]{103}; \& \node[draw]{0}; \\
      \node[draw]{104}; \& \node[draw]{1}; \\
      \node[draw]{111}; \& \node[draw]{0}; \\
    }

    % Fenwick tree for values 9-12.
    \FenwickTree{val-9-12.south}{
      \node[draw]{100}; \& \node[draw]{1}; \\
      \node[draw]{102}; \& \node[draw]{1}; \\
      \node[draw]{103}; \& \node[draw]{0}; \\
      \node[draw]{104}; \& \node[draw]{1}; \\
      \node[draw]{105}; \& \node[draw]{1}; \\
      \node[draw]{106}; \& \node[draw]{1}; \\
      \node[draw]{108}; \& \node[draw]{1}; \\
      \node[draw]{109}; \& \node[draw]{1}; \\
      \node[draw]{110}; \& \node[draw]{1}; \\
      \node[draw]{111}; \& \node[draw]{0}; \\
      \node[draw]{112}; \& \node[draw]{1}; \\
      \node[draw]{113}; \& \node[draw]{1}; \\
    }
  \end{tikzpicture}

  % \caption{}
  \label{fig:fenwick-tree-medwalk}
\end{figure}
