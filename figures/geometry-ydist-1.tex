\begin{figure}[H]
  \centering

  \begin{tikzpicture}[
    scale=0.8,
    every node/.style={scale=0.8},
    label distance=3mm,
    point/.style = { circle, fill, inner sep=2.5pt },
    ]
    % coordinate system
    \draw[->] (-0.2,0) -- (10,0) node[right]{$x$};
    \draw[->] (0,-0.2) -- (0,8) node[above]{$y$};
    \node at (0,0) [below left=2mm] {$O$};

    % ray R and point P
    \draw (0,0) -- (10,3) node[above right]{$R$};
    \coordinate[label=below:$P$] (P) at (8,4.4);
    \node at (P)[point] {};

    % dashed lines through P
    \draw[dashed] (0,2) -- (10,5);
    \draw[dashed] (0,4.4) -- (P);
    \draw[dashed] (P) -- (8,8);

    % areas where no candidate points can exist
    \fill [pattern={north east lines}, pattern color=black!30]
    (0,4.4) rectangle (8,8);
    \fill [pattern={vertical lines}, pattern color=black!30]
    (8,4.4) -- (10,5) -- (10,8) -- (8,8);
    \fill [pattern={north west lines}, pattern color=black!30]
    (0,0) -- (10,3) -- (10,5) -- (0,2);

    % other candidate points
    \coordinate[label=below right:$P'$] (P2) at (5,3.8);
    \node at (P2)[point] {};
    \coordinate[label=below left:$P''$] (P3) at (3,3.6);
    \node at (P3)[point] {};
    \draw (P) -- (P2) -- (P3);
    \draw[dashed] (P3) -- (2, 3.55);

    % bad points which can never be the answer due to P's existence
    \coordinate[label=above:$A$] (A) at (9,6);
    \node at (A)[point] {};
    \coordinate[label=above:$B$] (B) at (6,6);
    \node at (B)[point] {};
  \end{tikzpicture}

  \caption{}
  \label{fig:geometry-ydist-1}
\end{figure}
