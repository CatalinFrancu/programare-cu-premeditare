\begin{figure}[H]
  \centering

  \begin{tikzpicture}[
    scale=0.7,
    every node/.style={scale=0.7},
    spanned/.style={fill=blue!20},
    ]
    \begin{scope}[
      every node/.style={
        circle,
        draw,
        inner sep=2pt,
        minimum size=20pt,
      },
      level distance=20mm,
      level 1/.style={sibling distance=70mm},
      level 2/.style={sibling distance=45mm},
      level 3/.style={sibling distance=20mm}
      ]
      \node {1}
      child {node {7}
        child {node {4}}
        child {node {9}}
      }
      child {node[spanned] {5}
        child {node[spanned] {11}
          child {node {2}}
          child {node[spanned] {14}}
          child {node {6}}
        }
        child {node[spanned] {3}
          child {node {13}
            child {node {10}}
          }
          child {node[spanned] {8}}
        }
      }
      child {node {12}};
    \end{scope}

    \begin{scope}[shift={(-10,-9.5)}]
      % First row: indices
      \matrix[right] (index) at (0, 0) [
      matrix of nodes,
      nodes={anchor=center, minimum size=20pt, color=black!50, font=\small},
      ] {
        1 & 2 & 3 & 4 & 5 & 6 & 7 & 8 & 9 & 10 & 11 & 12 & 13 & 14 &
        15 & 16 & 17 & 18 & 19 & 20 & 21 & 22 & 23 & 24 & 25 & 26 & 27 & 28\\
      };

      % Second row: Euler linearization
      \node[label] (label-abstract) at (-0.5, -0.75) {Euler};

      \matrix[right] (abstract) at (0, -0.75) [
      matrix of nodes,
      nodes={draw, anchor=center, minimum size=20pt},
      column sep=-\pgflinewidth,
      row sep=-\pgflinewidth
      ] {
        1 & 7 & 4 & 4 & 9 & 9 & 7 & 5 & 11 & 2 & 2 & 14 & \node[spanned]{14}; &
        \node[spanned]{6}; & \node[spanned]{6}; & \node[spanned]{11}; & \node[spanned]{3}; &
        \node[spanned]{13}; & \node[spanned]{10}; & \node[spanned]{10}; & \node[spanned]{13}; &
        \node[spanned]{8}; & 8 & 3 & 5 & 12 & 12 & 1\\
      };
    \end{scope}
  \end{tikzpicture}

  \caption{Un arbore cu liniarizarea Euler și intervalul corespunzător interogării pe calea $(14, 8)$.}
  \label{fig:tree-mo}
\end{figure}
