\begin{figure}[H]
  \centering

  \begin{tikzpicture}[
    node/.style={
      circle,
      draw=black,
      minimum size=0.75cm,
      thick,
    },
    subtree/.style={
      draw=black,
      minimum size=1.75cm,
      regular polygon sides=3,
      regular polygon,
    }
    ]

    \node[node] (u) at (0,0) {$u$};
    \node[node] (v) at (5,0) {$v$};
    \node[node] (w) at (10,0) {$w$};

    \node[subtree] (s1) at (-0.9,-3) {};
    \node[subtree] (s2) at (0.9,-3) {};
    \node[subtree] (s3) at (4.1,-3) {};
    \node[subtree] (s4) at (5.9,-3) {};
    \node[subtree] (s5) at (8.2,-3) {};
    \node[subtree] (s6) at (10,-3) {};
    \node[subtree] (s7) at (11.8,-3) {};

    \draw (u) -- (v) -- (w);
    \draw (s1.north) -- (u);
    \draw (s2.north) -- (u);
    \draw (s3.north) -- (v);
    \draw (s4.north) -- (v);
    \draw (s5.north) -- (w);
    \draw (s6.north) -- (w);
    \draw (s7.north) -- (w);

    \draw[densely dotted,thick] (-2,1.5) rectangle (7,-4);
    \node at (-1.5, 1) {$S_2$};
    \draw[densely dotted,thick] (3,1) rectangle (13,-4.5);
    \node at (12.5, 0.5) {$S_1$};
  \end{tikzpicture}

  \caption{Un arbore ipotetic cu doi centroizi aflați la distanță 2.}
  \label{fig:tree-adjacent-centroids}
\end{figure}
