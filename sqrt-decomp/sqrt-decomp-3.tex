\section{Descompunere în valori distincte}

Vom da două exemple de probleme care ating timp $\bigoh(n \sqrt{n})$ dintr-un motiv matematic. Tehnica nu duce la descompunere în radical „convențională”, cu blocuri, dar nu știu unde altundeva să o încadrez.

Observația este: dacă suma unor numere naturale este $n$, atunci numerele au $\bigoh(\sqrt{n})$ valori distincte. Demonstrația decurge din inversa sumei Gauss.
